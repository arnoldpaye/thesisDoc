\chapter{Marco Referencial}
\section{Introducci\'on}
A trav\'es de la historia, el conocimiento ha sido almacenado en gram medida de forma
escrita, teni\'endose a los libros, peri\'odicos, art\'iculos cient\'ificos, etc.,
como muestra de ello. En general este conocimiento ha sido escrito en lenguaje
natural.\\

Dada la abrumadora cantidad de este tipo de informaci\'on, surge la necesidad de su
organizaci\'on para su correcta identificaci\'on entre las diferentes \'areas de
conocimiento a las que puede pertenecer a fin de facilitar su consuta e indexaci\'on;
para este cometido es necesario contar con criterios de clasificaci\'on que faciliten
su posterior consulta.\\

El Procesamiento del Lenguaje Natural (PLN), es una rama de la Ling\"u\'istica
Computacional que b\'asicamente estudia la comunicaci\'on entre personas y m\'aquinas
por medio de lenguajes naturales. Entre las principales tareas de las que se ocupa,
se encuentra la \emph{extracci\'on de informaci\'on} que es utilizada para el
reconocimiento de nombres de entidades y extracci\'on de terminolog\'ia. \\

La extracci\'on de terminolog\'ia permite identificar y seleccionar \emph{palabras
clave} de un texto analizado, que representan su tema o motivo central, las cuales
son un criterio de clasificaci\'on eficiente. \\

Existen dos enfoques claramente identificados para la selecci\'on automatizada de
palabras clave: en base a aprendizaje supervisado y no supervisado, en el primero
la computadora es previamente entrenada con un conjunto de textos y palabras clave
seleccionadas por un experto ling\"uista para su posterior automatizaci\'on, mientras
que en el segundo la automatizaci\'on es totalmente independiente. \\

En este trabajo realizaremos la selecci\'on de palabras clave mediante un enfoque
no supervisado mediante el modelo \emph{TextRank} que utiliza algoritmos de
clasificaci\'on basados en grafos.

\section{Antecedentes}
\section{Planteamiento del problema}
\section{Objetivos}
\section{Justificativos}
\section{Aportes}
\section{L\'imites y alcanes}
\section{Metodolog\'ia}
